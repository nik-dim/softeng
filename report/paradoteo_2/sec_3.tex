\section{Προδιαγραφές απαιτήσεων λογισμικού}

\subsection{Εξωτερικές διεπαφές}
Λεπτομερής τεχνική προδιαγραφή των διεπαφών που αναφέρονται στην ενότητα 1.3.1.
Προαιρετική χρήση λογισμικού προτυποποίησης διεπαφών χρήστη (mock-up).

\subsection{Λειτουργίες: περιπτώσεις χρήσης}
Λεπτομερής προδιαγραφή των λειτουργιών του λογισμικού σε επίπεδο περιπτώσεων χρήσης.
Για κάθε μία λειτουργία δίνονται τα ακόλουθα.
ΟΜΑΔΕΣ 5 ΑΤΟΜΩΝ: 2-3 περιπτώσεις χρήσης
ΟΜΑΔΕΣ 6 ΑΤΟΜΩΝ: 3-4 περιπτώσεις χρήσης
ΟΜΑΔΕΣ 7 ΑΤΟΜΩΝ: 4-5 περιπτώσεις χρήσης


\subsection{Εξωτερικές διεπαφές}
Σε αυτή την ενότητα θα περιγράψουμε τις τεχνικές προδιαγραφές των εξωτερικών διεπαφών.\\ \hfill \\
Όσον αφορά το MapBox API, για να μπορέσει να αποτυπώσει σωστά τα καταστήματα στο χάρτη θα χρειαστεί να παρέχουμε τις πληροφορίες με ένα εξειδικευμένο τύπο \texttt{JSON}, τον οποίο το MapBox αποκαλεί \texttt{GEOJSON}. Αυτά είναι της μορφής:
\begin{center}
    \begin{lstlisting}
        {
          "type": "Feature",
          "geometry": {
                        "type": "Point",
                        "coordinates": [lng, log]
                      },
          "properties": {
                          "name": name
                        }
        }
        \end{lstlisting}
\end{center}

Έπειτα, το REST API βασίζεται στο δομότυπο \texttt{JSON}, με τον οποίο η επικοινωνία Frontend και Backend γίνεται πιο εύκολη. Πιο συγκεκριμένα το Frontend κάνει συγκεκριμένα \texttt{HTTP} αιτήματα (στα οποία για τις περιπτώσεις PUT / POST περιέχουν και \texttt{JSON} αντικείμενα) στον Server του Backend, ο οποίος απαντά με ένα \texttt{JSON Response}. Η μορφή ενός τέτοιου μηνύματος είναι η εξής:

\begin{center}
    \begin{lstlisting}
        {
          "name": name,
          "lng": lng,
          "lat": lat,
          "brand": brand,
          ...
        }
        \end{lstlisting} 
\end{center}
Επιπλέον, για να βεβαιωθούμε ότι κάποιες λειτουργίες μπορούν να γίνουν μόνο από εγγεγραμμένο χρήστη, ενώ κάποιες άλλες μόνο από διαχειριστή, χρησιμοποιούμε \texttt{JWT} Authentication. Αυτό δημιουργεί σε κάθε επιτυχημένο login ένα token το οποίο επιστρέφεται μία φορά κατά το login στο Frontend, το οποίο είναι υποχρεωμένο να το κρατήσει και να το στέλνει στην επικεφαλίδα των \texttt{HTTP} αιτημάτων.
\subsubsection{ΠΕΡΙΠΤΩΣΗ ΧΡΗΣΗΣ 2: (τίτλος)}
\paragraph{Χρήστες (ρόλοι) που εμπλέκονται}
Αναφορά στους ρόλους που αφορά η περίπτωση χρήσης
\paragraph{Προϋποθέσεις εκτέλεσης}
Καταγραφή των συνθηκών που πρέπει να ισχύουν ώστε να μπορεί να εκτελεστεί η περίπτωση χρήσης
\paragraph{Περιβάλλον εκτέλεσης}
Αναφορά στο περιβάλλον στο οποίο εκτελείται η περίπτωση χρήσης. Πχ "διαδικτυακή διεπαφή χρήστη",
"DBMS" κλπ
\paragraph{Δεδομένα εισόδου}
Καταγραφή δεδομένων εισόδου και εξόδου και συνθηκών εγκυρότητας αυτών.
\paragraph{Παράμετροι}
Καταγραφή παραμέτρων και συνθηκών εγκυρότητας αυτών
\paragraph{Αλληλουχία ενεργειών - επιθυμητή συμπεριφορά}
Περιγραφή με κείμενο (Βήμα 1, Βήμα 2 κλπ) και διαγράμματα UML αλληλουχίας (Sequence) και
δραστηριοτήτων (Activity). Περιλαμβάνεται η συμπεριφορά σε απρόβλεπτες καταστάσεις και σφάλματα
(εναλλακτικές ροές).
\paragraph{Δεδομένα εξόδου}
Διαγράμματα UML αλληλουχίας για την παραγωγή δεδομένων εξόδου. Ως δεδομένα εξόδου νοούνται όλα τα
δεδομένα του συστήματος τα οποία δημιουργούνται ή μεταβάλλονται κατά την εκτέλεση (αν υπάρχουν
τέτοια)
\paragraph{Παρατηρήσεις}
Ο,τι δεν εντάσσεται στα προηγούμενα, εφόσον υπάρχει


\subsection{Απαιτήσεις επιδόσεων}
Ποσοτική τεκμηρίωση μέτρων και κριτηρίων επιθυμητών επιδόσεων με αναφορά στα ποσοτικά
χαρακτηριστικά εισόδων και φορτίου του λογισμικού.
\subsection{Απαιτήσεις οργάνωσης δεδομένων}
\subsubsection{Τεχνική περιγραφή των δεδομένων που διαχειρίζεται το λογισμικό και των σχετικών
μετρικών φορτίου δεδομένων εισόδου, επεξεργασίας κ.λπ.}
Αναλυτική αναφορά στα δεδομένα εισόδου, τα σχετικά πρότυπα δεδομένων και υπηρεσιών, καθώς και σε
μετρικές που σχετίζονται με τα δεδομένα (storage capacity planning).
\subsubsection{Απαιτήσεις και περιορισμοί πρόσβασης σε δεδομένα}
Απαιτήσεις πρόσβασης και περιορισμοί.
(ΤΑΥΤΟΤΗΤΑ ΟΜΑΔΑΣ) ΕΓΓΡΑΦΟ SRS Σελ 4 / 4
\subsubsection{Μοντέλο δεδομένων (μοντέλο κλάσεων UML ή/και μοντέλο ER)}
Μοντέλα δεδομένων UML ή/και ER
\subsubsection{Προδιαγραφές ακεραιότητας δεδομένων}
Κανόνες ακεραιότητας και εγκυρότητας δεδομένων
\subsubsection{Προδιαγραφές διατήρησης δεδομένων}
Απαιτήσεις διατήρησης δεδομένων σε βάθος χρόνου.
\subsection{Περιορισμοί σχεδίασης}
\textbf{Λεπτομερής τεχνική τεκμηρίωση των περιορισμών σχεδίασης οι οποίοι επιβάλλονται από απαιτήσεις συμμόρφωσης σε πρότυπα, κανονισμούς, ή άλλους περιορισμούς του έργου. Περιλαμβάνεται η πολιτική
ονοματολογίας οντοτήτων δεδομένων και πεδίων. Τέτοιοι περιορισμοί μπορεί να επιβάλλονται από τη χρήση βιβλιοθηκών, frameworks, περιβαλλόντων ανάπτυξης κλπ} \\
Αρχικά αναφέρουμε τα σχεδιαστικά εργαλεία που χρησιμοποιούμε. Για την υλοποποίηση του backend χρησιμοποιήσαμε την \texttt{JavaScript} με το \texttt{NodeJS} framework. Η επιλογή αυτού του framework έγινε διότι είναι αρκετά διαδεδομένο για την ευελιξία που παρέχει στον προγραμματιστή. Για build automation χρησιμοποιούμε το \texttt{npm}. Το εργαλείο που χρησιμοποιείται για την επικοινωνία του backend με τη βάση δεδομένων είναι το \texttt{MongoDB}. Σημειώνουμε επίσης ότι επιλέχθηκε και ο online server που παρέχει αυτό το εργαλείο. Τέλος, για τη δυναμική υλοποίηση του frontend, χρησιμοποιούμε μέσω της \texttt{Angular} \texttt{TypeScript}, \texttt{HTML} και \texttt{CSS}.\\
Όσον αφορά το backend, για την ονοματολογία των οντοτήτων χρησιμοποιούμε κεφαλαίο στο πρώτο γράμμα (π.χ. Price, User, κτλ) και για τις συναρτήσεις μικρό το πρώτο γράμμα, ενώ όταν υπάρχουν πολλές λέξεις, η κάθε μια ξεκινά με κεφαλαίο γράμμα (π.χ. Price.findById).
Αντίστοιχα, στο frontend η μόνη διαφορά σε σχέση με πριν είναι ότι τα ονόματα των οντοτήτων ξεκινούν με μικρό γράμμα. \\
Η χρήση του εργαλείου \texttt{MongoDB} για την επικοινωνία του backend με τη βάση δεδομένων περιέχει τον περιορισμό των 512 MB storage, καθώς επιλέξαμε το free plan για το εργαλείο αυτο. Ανάλογα βέβαια με την επιτυχία της εφαρμογής μας, μπορούμε αν χρειαστεί να χρησιμοποιήσουμε κάποιο αναβαθμισμένο plan του εργαλείου αυτού, προκειμένου να επεκταθεί το μέγεθος του storage \\
Σχετικά με τη συμμόρφωση σε πρότυπα, για τις συντεταγμένες των χρηστών και των καταστημάτων χρησιμοποιήθηκε το σύστημα συντεταγμένων \texttt{WGS84}. Επιπλέον, οι κωδικοί για το login των χρηστών αποθηκεύονται σε hashed μορφή, το οποίο εξασφαλίζει ότι είναι κωδικοποιημένοι.

link: https://www.mongodb.com/cloud/atlas/pricing

\subsection{Λοιπές απαιτήσεις}
\subsubsection{Απαιτήσεις διαθεσιμότητας λογισμικού}
Τεκμηρίωση απαιτήσεων διαθεσιμότητας
\subsubsection{Απαιτήσεις ασφάλειας}
Τεκμηρίωση απαιτήσεων ασφαλείας
\subsubsection{Απαιτήσεις συντήρησης}
Τεκμηρίωση απαιτήσεων συντήρησης