\subsubsection{ΠΕΡΙΠΤΩΣΗ ΧΡΗΣΗΣ 2: (τίτλος)}
\paragraph{Χρήστες (ρόλοι) που εμπλέκονται}
Αναφορά στους ρόλους που αφορά η περίπτωση χρήσης
\paragraph{Προϋποθέσεις εκτέλεσης}
Καταγραφή των συνθηκών που πρέπει να ισχύουν ώστε να μπορεί να εκτελεστεί η περίπτωση χρήσης
\paragraph{Περιβάλλον εκτέλεσης}
Αναφορά στο περιβάλλον στο οποίο εκτελείται η περίπτωση χρήσης. Πχ "διαδικτυακή διεπαφή χρήστη",
"DBMS" κλπ
\paragraph{Δεδομένα εισόδου}
Καταγραφή δεδομένων εισόδου και εξόδου και συνθηκών εγκυρότητας αυτών.
\paragraph{Παράμετροι}
Καταγραφή παραμέτρων και συνθηκών εγκυρότητας αυτών
\paragraph{Αλληλουχία ενεργειών - επιθυμητή συμπεριφορά}
Περιγραφή με κείμενο (Βήμα 1, Βήμα 2 κλπ) και διαγράμματα UML αλληλουχίας (Sequence) και
δραστηριοτήτων (Activity). Περιλαμβάνεται η συμπεριφορά σε απρόβλεπτες καταστάσεις και σφάλματα
(εναλλακτικές ροές).
\paragraph{Δεδομένα εξόδου}
Διαγράμματα UML αλληλουχίας για την παραγωγή δεδομένων εξόδου. Ως δεδομένα εξόδου νοούνται όλα τα
δεδομένα του συστήματος τα οποία δημιουργούνται ή μεταβάλλονται κατά την εκτέλεση (αν υπάρχουν
τέτοια)
\paragraph{Παρατηρήσεις}
Ο,τι δεν εντάσσεται στα προηγούμενα, εφόσον υπάρχει
