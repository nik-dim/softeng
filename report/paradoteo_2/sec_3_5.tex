
\subsection{Περιορισμοί σχεδίασης}
% \textbf{Λεπτομερής τεχνική τεκμηρίωση των περιορισμών σχεδίασης οι οποίοι επιβάλλονται από απαιτήσεις συμμόρφωσης σε πρότυπα, κανονισμούς, ή άλλους περιορισμούς του έργου. Περιλαμβάνεται η πολιτική
% ονοματολογίας οντοτήτων δεδομένων και πεδίων. Τέτοιοι περιορισμοί μπορεί να επιβάλλονται από τη χρήση βιβλιοθηκών, frameworks, περιβαλλόντων ανάπτυξης κλπ} \\

Αρχικά αναφέρουμε τα σχεδιαστικά εργαλεία που χρησιμοποιούμε. Για την υλοποποίηση του backend χρησι-μοποιούμε την \texttt{JavaScript} με το \texttt{NodeJS} framework. Η επιλογή αυτού του framework έγινε διότι είναι αρκετά διαδεδομένο για την ευελιξία που παρέχει στον προγραμματιστή. Για build automation χρησιμοποιούμε το \texttt{npm}. Το εργαλείο που χρησιμοποιείται για την επικοινωνία του backend με τη βάση δεδομένων είναι το \texttt{MongoDB}. Σημειώνουμε επίσης ότι επιλέχθηκε και ο online server που παρέχει αυτό το εργαλείο. Τέλος, για τη δυναμική υλοποίηση του frontend, χρησιμοποιούμε \texttt{TypeScript}, \texttt{HTML} και \texttt{CSS} μέσω της \texttt{Angular} .\\
Όσον αφορά το backend, για την ονοματολογία των οντοτήτων χρησιμοποιούμε κεφαλαίο στο πρώτο γράμμα (π.χ. Price, User, κτλ) και για τις συναρτήσεις μικρό το πρώτο γράμμα, ενώ όταν υπάρχουν πολλές λέξεις, η κάθε μια ξεκινά με κεφαλαίο γράμμα (π.χ. Price.findById).
Αντίστοιχα, στο frontend η μόνη διαφορά σε σχέση με πριν είναι ότι τα ονόματα των οντοτήτων ξεκινούν με μικρό γράμμα. \\
Η χρήση του εργαλείου \texttt{MongoDB} για την επικοινωνία του backend με τη βάση δεδομένων περιέχει τον περιορισμό των 512 MB storage, καθώς επιλέξαμε το free plan για το εργαλείο αυτο. Ανάλογα βέβαια με την επιτυχία της εφαρμογής μας, μπορούμε, αν χρειαστεί, να χρησιμοποιήσουμε κάποιο αναβαθμισμένο plan του εργαλείου αυτού, προκειμένου να επεκταθεί το μέγεθος του storage. \\
Σχετικά με τη συμμόρφωση σε πρότυπα, για τις συντεταγμένες των χρηστών και των καταστημάτων χρησιμο-ποιείται το σύστημα συντεταγμένων \texttt{WGS84}. Επιπλέον, οι κωδικοί για το login των χρηστών αποθηκεύονται σε hashed μορφή, το οποίο εξασφαλίζει ότι είναι κωδικοποιημένοι.