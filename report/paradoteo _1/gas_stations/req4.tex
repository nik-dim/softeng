\section{Λειτουργικές απαιτήσεις επιχειρησιακού περιβάλλοντος}



\subsection{Επιχειρησιακές διαδικασίες}

\textit{Ροές εργασιών κατά τη συλλογή και πρόσβαση σε δεδομένα της εφαρμογής μας}

\begin{enumerate}
	\item Είναι υποχρεωμένος να φτιάξει προφίλ.
	\item Κάνει log-in.
	\item Στη συνέχεια έχει πρόσβαση σε λίστες με δυνατότητα ταξινόμησης,φίλτρων,Αναζήτηση προϊόντων και καταστημάτων
\end{enumerate}

\subsection{Περιορισμοί}
Δεν έχουν κάποιον συγκεκριμένο περιορισμό αφού θεωρούνται σαν απλοί χρήστες.

\subsection{Δείκτες ποιότητας}

\begin{enumerate}
	\item Χρησιμότητα
	\begin{enumerate}
		\item Εύκολο στην χρήση και στην κατανόηση.
		\item Εύκολο στο να βρίσκεις πως να χρησιμοποιείς μηχανές αναζήτησης.
	\end{enumerate}
	\item Αξιοπιστία
	\begin{enumerate}
		\item Εύκολο να θυμάσαι το URL
		\item Λιγότερες διαφημίσεις 
	\end{enumerate}
	\item Διαδραστικά χαρακτηριστικά
	\begin{enumerate}
		\item FAQ
		\item Feedback μεταξύ χρήστη και website (chat)
	\end{enumerate}
	\item Mapping
	\begin{enumerate}
		\item Επαρκές website map ή navigation menu
	\end{enumerate}
	\item Links
	\begin{enumerate}
		\item Βοηθητικοί σύνδεσμοι(π.χ πίσω στο Home Page)
		\item Αξιόλογοι σύνδεσμοι(π.χ σε άλλα παρόμοια websites,όχι dead links)
	\end{enumerate}
	\item Design
	\begin{enumerate}
		\item Καινοτόμο 
		\item Οι χρησιμοποιούμενες εικόνες να υπηρετούν κάποιο λειτουργικό σκοπό
		\item Σταθερότητα στο style
		\item Ικανοποιητικό spacing
		\item Δυνατότητα scrolling
	\end{enumerate}
	\item Περιεχόμενο
	\begin{enumerate}
		\item Up-to-date πληροφορίες
		\item Συχνή ενημέρωση του website
		\item Επικυρωμένες πηγές πληροφοριών
		\item Όχι γραμματικά και συντακτικά λάθη
		\item Αναφορά στη φυσική διεύθυνση του οργανισμού
		\item Αναφορά στα copyrights
	\end{enumerate}