\section{Διαχειριστικές απαιτήσεις επιχειρησιακού περιβάλλοντος}

\subsection{Επιχειρησιακό μοντέλο}
Ο λόγος για τον οποίο η εφαρμογή θα είναι επιτυχημένη από την πλευρά των πρατηρίων βενζίνης είναι η παρουσία τους στην αγορά σε σχέση με τους ανταγωνιστές τους.
Πιο συγκεκριμένα,επιδιώκουν μέσω της εφαρμογής μας κάνουν ανάλυση του ανταγωνισμού ,να μαθαίνουν τις τιμές των άλλων πρατηρίων.Με τον τρόπο αυτό θα μπορούν να προσαρμόζονται κατάλληλα στις ανάγκες της αγοράς ,τοποθετώντας και τις δικές τους τιμές σε ένα λογικό πλαίσιο.. 


\subsection{Περιβάλλον διαχείρισης πληροφοριών}
Η εικόνα διαχείρισης πληροφοριών σήμερα στο τομέα των υγρών καυσίμων  εκφράζεται κυρίως μέσω των παρακάτω ανταγωνιστών μας.
Αρχικά,υπάρχει το website http://www.fuelprices.gr ,το οποίο παρέχεται μέσω του Υπουργείου Ανάπτυξης,Ανταγωνιστικότητας,Υποδομών,Μεταφορών και Δικτύων.Αποτελεί ουσιαστικά ένα παρατηρητήριο τιμών για όλη την ελληνική επικράτεια όπου όμως οι τιμές εισάγονται από τους ίδιους τους κατόχους πρατηρίων.
Επίσης,υπάρχει σαν εφαρμογή το Fuelio ,το οποίο ασχολείται γενικότερα με τα έξοδα αυτοκινήτου ,οπως κόστη service ,αλλά ταυτόχρονα κάνοντας crowdsourcing στις τιμές υγρών καυσίμων μας καθιστά ανταγωνιστές.
