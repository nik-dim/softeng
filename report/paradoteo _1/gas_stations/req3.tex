\section{Διαχειριστικές απαιτήσεις επιχειρησιακού περιβάλλοντος}

\subsection{Επιχειρησιακό μοντέλο}
Ο λόγος για τον οποίο η εφαρμογή θα είναι επιτυχημένη από την πλευρά των ιδιοκτητών πρατηρίων υγρών καυσίμων είναι η παρουσία τους στην αγορά σε σχέση με τους ανταγωνιστές τους.
Πιο συγκεκριμένα, επιδιώκουν μέσω της εφαρμογής μας να αναλύσουν τον ανταγωνισμό και την αγορά μαθαίνοντας τις τιμές των άλλων πρατηρίων. Με τον τρόπο αυτό θα μπορούν να προσαρμόζονται κατάλληλα στις ανάγκες της αγοράς, επιλέγοντας τη δική τους επιχειρηματική στρατηγική.


\subsection{Περιβάλλον διαχείρισης πληροφοριών}
Η εικόνα διαχείρισης πληροφοριών σήμερα στο τομέα των υγρών καυσίμων  εκφράζεται κυρίως μέσω των παρακάτω.
Αρχικά, υπάρχει το website http://www.fuelprices.gr, το οποίο παρέχεται μέσω του Υπουργείου Ανάπτυξης, Ανταγωνιστικότητας, Υποδομών, Μεταφορών και Δικτύων. Αποτελεί ουσιαστικά ένα παρατηρητήριο τιμών για την ελληνική επικράτεια όπου όμως οι τιμές εισάγονται από τους ίδιους τους ιδιοκτήτες πρατηρίων. 
Επίσης, υπάρχει σαν εφαρμογή το fuel.io, που δραστηριοποιείται σε αγορά του εξωτερικού και ασχολείται γενικότερα με τα έξοδα αυτοκινήτου, οπως κόστη service. Εκτός αυτών, μέσω crowdsoursing πλατφόρμας παρέχει πληροφορίες για τις τρέχουσες τιμές καυσίμων.