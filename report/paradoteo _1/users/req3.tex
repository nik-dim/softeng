\section{Διαχειριστικές απαιτήσεις επιχειρησιακού περιβάλλοντος}

\subsection{Επιχειρησιακό μοντέλο}
Ο λόγος για τον οποίο η εφαρμογή θα γίνει επιτυχημένη από την πλευρά των χρηστών είναι σαφώς η εξοικονόμηση χρημάτων που θα κάνουν μέσω αυτής.
Πιο συγκεκριμένα,οι χρήστες επιδιώκουν να βρουν τις χαμηλότερες τιμές μέσω της εφαρμογής μας.Βλέποντας τις τιμές αλλά και την τοποθεσία του θα μπορεί να βρει τα πιο κοντινά πρατήρια βενζίνης με τις τρέχουσες καλύτερες τιμές.Με τον τρόπο αυτό σε μια εποχή μείζονος σημασίας για τα οικονομικά του κάθε πολίτη θα αποτελέσει ένα βοήθημα για την ελαχιστοποίηση του κόστους μεταφοράς του.


\subsection{Περιβάλλον διαχείρισης πληροφοριών}
Η εικόνα διαχείρισης πληροφοριών σήμερα στο τομέα των υγρών καυσίμων  εκφράζεται κυρίως μέσω των παρακάτω ανταγωνιστών μας.
Αρχικά,υπάρχει το website http://www.fuelprices.gr ,το οποίο παρέχεται μέσω του Υπουργείου Ανάπτυξης,Ανταγωνιστικότητας,Υποδομών,Μεταφορών και Δικτύων.Αποτελεί ουσιαστικά ένα παρατηρητήριο τιμών για όλη την ελληνική επικράτεια όπου όμως οι τιμές εισάγονται από τους ίδιους τους κατόχους πρατηρίων.
Επίσης,υπάρχει σαν εφαρμογή το Fuelio ,το οποίο ασχολείται γενικότερα με τα έξοδα αυτοκινήτου ,οπως κόστη service ,αλλά ταυτόχρονα κάνοντας crowdsourcing στις τιμές υγρών καυσίμων μας καθιστά ανταγωνιστές.


