\section{Διαχειριστικές απαιτήσεις επιχειρησιακού περιβάλλοντος}

\subsection{Επιχειρησιακό μοντέλο}
Ο λόγος για τον οποίο η εφαρμογή θα γίνει επιτυχημένη από την πλευρά των χρηστών είναι η εξοικονόμηση χρημάτων που θα κάνουν μέσω αυτής.
Πιο συγκεκριμένα,οι χρήστες επιδιώκουν να βρουν τις χαμηλότερες τιμές μέσω της εφαρμογής μας. Εφόσον οι χρήστες παρέχουν την τοποθεσία τους, το σύστημα θα μπορεί να τους ενημερώσει για τα κοντονότερα πρατήρια, καθώς και τις χαμηλότερες τιμές υγρών καυσίμων πλησίον τους.Με τον τρόπο αυτό σε μια εποχή μείζονος σημασίας για τα οικονομικά του κάθε πολίτη θα αποτελέσει ένα βοήθημα για την ελαχιστοποίηση του κόστους μεταφοράς του.


\subsection{Περιβάλλον διαχείρισης πληροφοριών}
Η εικόνα διαχείρισης πληροφοριών σήμερα στο τομέα των υγρών καυσίμων  εκφράζεται κυρίως μέσω των παρακάτω.
Αρχικά, υπάρχει το website http://www.fuelprices.gr, το οποίο παρέχεται μέσω του Υπουργείου Ανάπτυξης, Ανταγωνιστικότητας, Υποδομών, Μεταφορών και Δικτύων. Αποτελεί ουσιαστικά ένα παρατηρητήριο τιμών για την ελληνική επικράτεια όπου όμως οι τιμές εισάγονται από τους ίδιους τους ιδιοκτήτες πρατηρίων. 
Επίσης, υπάρχει σαν εφαρμογή το fuel.io, που δραστηριοποιείται σε αγορά του εξωτερικού και ασχολείται γενικότερα με τα έξοδα αυτοκινήτου, οπως κόστη service. Εκτός αυτών, μέσω crowdsoursing πλατφόρμας παρέχει πληροφορίες για τις τρέχουσες τιμές καυσίμων.


