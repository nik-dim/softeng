\section{Διαχειριστικές απαιτήσεις επιχειρησιακού περιβάλλοντος}

\subsection{Επιχειρησιακό μοντέλο}
Η εφαρμογή που σχεδιάζουμε θα είναι χρήσιμη και αρκετά διαδεδομένη καθώς, αρχικά, αφορά ένα απαραίτητο προϊόν που η πλειοψηφία των ανθρώπων χρησιμοποιεί καθημερινά. Επίσης, οι τιμές από πρατήριο σε πρατήριο διαφέρουν και τα τελευταία χρόνια, γενικά, υπάρχει αύξηση τιμών στα υγρά καύσιμα. Οπότε ο καταναλωτής-χρήστης έχει σοβαρό κίνητρο να ασχοληθεί προς όφελός του.



\subsection{Περιβάλλον διαχείρισης πληροφοριών}
Ως administrators, για την διαχείριση πληροφοριών χρησιμοποιούμε ένα DBMS το οποίο συνδέεται μέσω java με την τεχνολογία gradle (backend) όπου με την σειρά του ανταλλάσσει δεδομένα μέσω javascript με το frontend της εφαρμογής μας (χρήστες βλέπουν-καταγράφουν τιμές).



