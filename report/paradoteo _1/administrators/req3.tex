\section{Διαχειριστικές απαιτήσεις επιχειρησιακού περιβάλλοντος}

\subsection{Επιχειρησιακό μοντέλο}
Η εφαρμογή που σχεδιάζουμε θα είναι χρήσιμη και αρκετά διαδεδομένη καθώς, αρχικά, αφορά ένα απαραίτητο προϊόν που η πλειοψηφία των ανθρώπων χρησιμοποιεί καθημερινά. Επίσης, οι τιμές από πρατήριο σε πρατήριο διαφέρουν και τα τελευταία χρόνια, γενικά, υπάρχει αύξηση τιμών στα υγρά καύσιμα. Οπότε ο καταναλωτής-χρήστης έχει σοβαρό κίνητρο να ασχοληθεί προς όφελός του,γιατί το να εξοικονομεί χρήματα σε καθημερινή βάση για τις μεταφορές του στην εποχή της κρίσης είναι μείζονος σημασίας. Αυτό θα αυξήσει τη δημοτικότητα μας και  θα μπορούμε να αποτελούμε διαφημιστικό μέσο.Επίσης ο ιδιώτης-ιδιοκτήτης πρατηρίου έχει κίνητρο να ασχοληθεί προς όφελός του λόγω του ανταγωνισμού που θα δημιουργηθεί,καθώς όποιος είναι μέλοςι στην εφαρμογή αυτόματα θα διαφημίζεται στους χρήστες εθελοντές.


\subsection{Περιβάλλον διαχείρισης πληροφοριών}



