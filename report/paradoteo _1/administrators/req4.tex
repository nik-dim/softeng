\section{Λειτουργικές απαιτήσεις επιχειρησιακού περιβάλλοντος}



\subsection{Επιχειρησιακές διαδικασίες}
\textit{Ροές εργασιών κατά τη συλλογή και πρόσβαση σε δεδομένα της εφαρμογής μας}


\begin{enumerate}
	\item Κάνει log in
	\item Στη συνέχεια έχει πρόσβαση σε λίστες με δυνατότητα ταξινόμησης,φίλτρων και αναζήτησης
	\item Κάνει έλεγχο τιμών,έτσι ώστε να κινούνται σε λογικά πλαίσια
	\item Αντίστοιχα έχει τη δυνατότητα διαχείρισης προφίλ
\end{enumerate}



\subsection{Περιορισμοί}
Ως administrators ένας περιορισμός είναι η πιθανότητα να μην έχουμε άδεια προβολής κάποιων πληροφοριών χρηστών η πρατηρίων πχ τιμές συγκεκριμένων καυσίμων, διευθύνσεις κατοικίας κλπ. Επίσης εφόσον εξαρτώμαστε από τις καταγραφές των εθελοντών-χρηστών, υπάρχει ο κίνδυνος έλλειψης αξιοπιστίας σε περίπτωση λάθους ή κακόβουλης εισαγωγής δεδομένων.


\subsection{Δείκτες ποιότητας}
\begin{enumerate}
	\item Συχνή επισκεψιμότητα
	\item Αριθμός εγγεγραμμένων χρηστών 
	\item Αξιοπιστία τιμών
\end{enumerate}
